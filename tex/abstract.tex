Bayesian methods offer a way of doing a fully probabilistic analysis for inference problems and for dealing with uncertainty. 
They offer a number of advantages over more conventional statistical techniques and have become widely used in the fields of genetics, genomics, bioinformatics and computational systems biology, in an effort to make sense of complex noisy data. 
Here, we describe an application area dealing with modeling spatial patterning in developing \textit{Drosophila} embryo.
In addition, after describing an analysis of association mapping in chickpea hybrids, we provide a perspective on how to integrate quantitative biology and systems biology approaches to increase power in association mapping studies.
Finally, I take a digression to describe a study on transposable element expression and chromatin accessibility in human cancer cells. 
