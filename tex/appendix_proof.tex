\chapter{Appendix}
\label{appendix}
\section{Population genomic analysis uncovers African and European admixture in \emph{Drosophila melanogaster} populations from the south-eastern United States and Caribbean Islands \cite{kao15}}

Drosophila melanogaster is postulated to have colonized North America in the past several 100 years in two waves. Flies from Europe colonized the east coast United States while flies from Africa inhabited the Caribbean, which if true, make the south- east US and Caribbean Islands a secondary contact zone for African and European D. melanogaster. This scenario has been proposed based on phenotypes and limited genetic data. In our study, we have sequenced individual whole genomes of flies from populations in the south-east US and Caribbean Islands and examined these popula- tions in conjunction with population sequences from the west coast US, Africa, and Europe. We find that west coast US populations are closely related to the European population, likely reflecting a rapid westward expansion upon first settlements into North America. We also find genomic evidence of African and European admixture in south-east US and Caribbean populations, with a clinal pattern of decreasing propor- tions of African ancestry with higher latitude. Our genomic analysis of D. melanogas- ter populations from the south-east US and Caribbean Islands provides more evidence for the Caribbean Islands as the source of previously reported novel African alleles found in other east coast US populations. We also find the border between the south- east US and the Caribbean island to be the admixture hot zone where distinctly Afri- can-like Caribbean flies become genomically more similar to European-like south-east US flies. Our findings have important implications for previous studies examining the generation of east coast US clines via selection.

\newpage

\section{Translating natural genetic variation to gene expression in a computational model of the \emph{Drosophila} gap gene regulatory network \cite{gursky2017}}
Annotating the genotype-phenotype relationship, and developing a proper quantitative description of the relationship, requires understanding the impact of natural genomic varia- tion on gene expression. We apply a sequence-level model of gap gene expression in the early development of Drosophila to analyze single nucleotide polymorphisms (SNPs) in a panel of natural sequenced D. melanogaster lines. Using a thermodynamic modeling frame- work, we provide both analytical and computational descriptions of how single-nucleotide variants affect gene expression. The analysis reveals that the sequence variants increase (decrease) gene expression if located within binding sites of repressors (activators). We show that the sign of SNP influence (activation or repression) may change in time and space and elucidate the origin of this change in specific examples. The thermodynamic modeling approach predicts non-local and non-linear effects arising from SNPs, and combi- nations of SNPs, in individual fly genotypes. Simulation of individual fly genotypes using our model reveals that this non-linearity reduces to almost additive inputs from multiple SNPs. Further, we see signatures of the action of purifying selection in the gap gene regulatory regions. To infer the specific targets of purifying selection, we analyze the patterns of poly- morphism in the data at two phenotypic levels: the strengths of binding and expression. We find that combinations of SNPs show evidence of being under selective pressure, while indi- vidual SNPs do not. The model predicts that SNPs appear to accumulate in the genotypes of the natural population in a way biased towards small increases in activating action on the expression pattern. Taken together, these results provide a systems-level view of how genetic variation translates to the level of gene regulatory networks via combinatorial SNP effects.


\newpage

\section{Genetic Diversity, Population Structure, and Genetic Correlation with Climatic Variation in Chickpea (\emph{Cicer arietinum}) Landraces from Pakistan \cite{Sani2018}}

Chickpea (Cicer arietinum L.) production in arid regions, such as those predominant in Pakistan, faces immense challenges of drought and heat stress. Addressing these challenges is made more difficult by the lack of genetic and phenotypic characterization of available cultivated varieties and breeding materials. Genotyping-by-sequencing offers a rapid and cost- effective means to identify genome-wide nucleotide variation in crop germplasm. When combined with extended crop phenotypes deduced from climatic variation at sites of collection, the data can predict which portions of genetic variation might have roles in climate resilience. Here we use 8113 single nucleotide polymorphism markers to determine genetic variation and compare population structure within a previously uncharacterized collection of 77 landraces and 5 elite cultivars, currently grown in situ on farms throughout the chickpea growing regions of Pakistan. The compiled landraces span a striking aridity gradient into the Thal Desert of the Punjab. Despite low levels of variation across the collection and limited genetic structure, we found some differentiation between accessions from arid, semiarid, irrigated, and coastal areas. In a subset of 232 markers, we found evidence of differentiation along gradients of elevation and isothermality. Our results highlight the utility of exploring large germplasm collections for nucleotide variation associated with environmental extremes, and the use of such data to nominate germplasm accessions with the potential to improve crop drought tolerance and other environmental traits.
